\documentclass[a4paper, 11pt]{article}

% Language and encoding.
\usepackage[english]{babel}
\usepackage[utf8]{inputenc}

% Set fonts (in order to deal with umlauts).
\usepackage[T1]{fontenc}
\usepackage{lmodern}

% Sets page size and margins.
\usepackage[a4paper, top=2.5cm, bottom=2.5cm, left=3cm, right=3cm, marginparwidth=1.75cm]{geometry}

% Useful packages
\usepackage{graphicx}
\usepackage{subcaption}
\usepackage{url}

% Lorem Ipsum package to generate dummy text. Simply remove it.
\usepackage{lipsum}

% Creative Commons license package.
\usepackage[
    type={CC},
    modifier={by-nc-sa},
    version={4.0},
]{doclicense}

% hyperref package to color links.
%\usepackage[colorlinks=true, allcolors=blue]{hyperref}

% Todo notes.
%\usepackage[textsize=footnotesize, backgroundcolor=yellow!10, linecolor=gray!35]{todonotes} 

\title{Seminar paper title}

\begin{document}

\date{\today}
\author{A. N. Other and A. N. Other}
\maketitle

\section{Introduction} \label{sec:intro}

To ensure best compatibility with version control (e.g.\ git), it is best to write one sentence per line and configure your editor not to hard-break lines.
For citations use \cite{BakSchaLewRotBla11} or \cite[p.\ 6--8]{BakSchaLewRotBla11}.
For quotes use, for example,
\begin{quote}
	``This is a proper quote.'' \cite[p.\ 6]{BakSchaLewRotBla11}
\end{quote}

\begin{figure}[t]
	\centering
	\includegraphics[width=0.25\textwidth]{figures/logo.png}
\caption{LaTeX logo.} \label{fig:logo}
\end{figure}

Sometimes, we would like to include figures.
This sentence provides an example of referencing to Fig.~\ref{fig:logo}.

You can of course also refer to Sec.~\ref{sec:intro} or to Sec.~\ref{sec:intro:motivation}.

Footnotes are includes, for example, here.\footnote{A footnote containing a link \url{https://free-and-open-technologies.github.io}}

% Creates dummy text. Simply remove.
\lipsum[1]

\subsection{Motivation} \label{sec:intro:motivation}

% Creates dummy text. Simply remove.
\lipsum[2-4]

% References section.
\bibliographystyle{plain}
\bibliography{bibliography}

% Delete sentence with emails if you don't want to be contacted.
\paragraph{About this document.} This seminar paper was written as part of the lecture \emph{Free and Open Technologies}, held by Christoph Derndorfer and Lukas~F.\ Lang at TU Wien, Austria, during the winter term 2019/2020.
The authors A.~N.~Other and A.~N.~Other can be contacted by email at \url{e9999999@student.tuwien.ac.at} and at \url{e9999999@student.tuwien.ac.at}.
All selected papers can be found online.\footnote{\url{https://free-and-open-technologies.github.io}}

% Add CC license.
\doclicenseThis

\end{document}
